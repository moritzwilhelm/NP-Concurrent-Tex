% Vorlage für eine Bachelorarbeit
% Siehe auch LaTeX-Kurs von Mathematik-Online
% www.mathematik-online.org/kurse
% Anpassungen für die Fakultät für Mathematik
% am KIT durch Klaus Spitzmüller und Roland Schnaubelt Dezember 2011

\documentclass[12pt,a4paper]{scrartcl}
% scrartcl ist eine abgeleitete Artikel-Klasse im Koma-Skript
% zur Kontrolle des Umbruchs Klassenoption draft verwenden


% die folgenden Packete erlauben den Gebrauch von Umlauten und ß
% in der Latex Datei
\usepackage[utf8]{inputenc}
% \usepackage[latin1]{inputenc} %  Alternativ unter Windows
\usepackage[T1]{fontenc}
\usepackage[ngerman]{babel}


\usepackage[pdftex]{graphicx}
\usepackage{latexsym}
\usepackage{amsmath,amssymb,amsthm}


% Abstand obere Blattkante zur Kopfzeile ist 2.54cm - 15mm
\setlength{\topmargin}{-15mm}


% Umgebungen für Definitionen, Sätze, usw.
% Es werden Sätze, Definitionen etc innerhalb einer Section mit
% 1.1, 1.2 etc durchnummeriert, ebenso die Gleichungen mit (1.1), (1.2) ..
\newtheorem{Satz}{Satz}[section]
\newtheorem{Definition}[Satz]{Definition} 
\newtheorem{Lemma}[Satz]{Lemma}		   
                  
\numberwithin{equation}{section} 

% einige Abkuerzungen
\newcommand{\C}{\mathbb{C}} % komplexe
\newcommand{\K}{\mathbb{K}} % komplexe
\newcommand{\R}{\mathbb{R}} % reelle
\newcommand{\Q}{\mathbb{Q}} % rationale
\newcommand{\Z}{\mathbb{Z}} % ganze
\newcommand{\N}{\mathbb{N}} % natuerliche



\begin{document}
  % Keine Seitenzahlen im Vorspann
  \pagestyle{empty}

  % Titelblatt der Arbeit
  \begin{titlepage}

     

 \begin{center} \large 
    
    Meilenstein A
    \vspace*{2cm}

    {\huge NP Projekt}
    \vspace*{2.5cm}
    
    
    \includegraphics[scale=0.45]{Answer_to_Life.png} 
    \vspace*{2cm}

	Daniel Emmel
    \vspace*{1.5cm}
    
    Moritz Wilhelm
    \vspace*{1.5cm}

    08.07.2019
    \vspace*{4.5cm}

		Universität des Saarlandes
  \end{center}
\end{titlepage}



  % Inhaltsverzeichnis
  \tableofcontents

\newpage
 


  % Ab sofort Seitenzahlen in der Kopfzeile anzeigen
  \pagestyle{headings}

\section{Meilenstein $ A $}

Nach der Aufgabenstellung wollen wir in diesem Projekt die gegebene sequentielle Implementierung in eine \textbf{nebenläufige}, äquivalente Umsetzung transformieren. \\
Dazu betrachten wir im Folgenden alle Stellen, an denen eine \textbf{nebenläufige} Umsetzung sinnvoll ist. Hierzu gehen wir den Programmfluss chronologisch beginnend bei $ Slug.main $ durch.\\\\
Wir sehen zunächst, dass in der Main die Bücher nacheinander abgearbeitet werden. Diesen Prozess können wir parallelisieren: \\
Dazu spawnen wir zusätzlich einen Thread pro Buch (minus 1), der dieses dann setzt, wobei der MainThread selbst auch eines verwerten kann.\\
Es ist hierbei zusätzlich zu beachten, dass der MainThread selbstverständlich am Ende das Programm terminieren muss. Dabei gibt es jedoch mehrere Fallstricke:\\
System.exit() terminiert alle Threads sofort. Daher könnte es sein, dass, falls der MainThread zuerst mit dem Setzen des ihm zugewiesenen Dokuments fertig ist, das ganze Programm beendet, obwohl die restlichen Threads noch am Verwerten ihrer jeweiligen Dokumente sind, also noch nicht terminieren sollten. \\
Folglich muss der MainThread abwarten, bis alle anderen Threads terminiert sind, also alle Bücher gedruckt wurden.  \\
Hierzu können wir einfach $ join() $ auf alle anfangs erzeugten Threads aufrufen.\\
Alternativ können wir beispielsweise einen globalen Integer verwenden, der angibt, wie viele der Dokumente setzenden Threads terminiert sind. Der MainThread wartet dann nach Fertigstellung des Setzens seines zugewiesenen Dokumentes bis ebenjener Integerwert mit der Anzahl der Bücher übereinstimmt. \\
Um sinnloses Busy-Wait zu verhindern, können wir selbstverständlich $ await() $ und $ signal() $ nutzen und den Integer locken müssen um Data Races zu verhindern.\\
\\
Nun betrachten wir die einzelne Dokumentensetzung entsprechend also $ creep() $:\\
Wir möchten das Einlesen des Buches, sprich das Parsen, nebenläufig zum Setzen des Dokumentes ausführen. Da jedoch sowohl der Parser als auch die spätere Verarbeitung der einzelnen Blockelemente auf die $ BlockElement $ Liste in $ Document $ zugreifen, müssen wir aus $ Document $ einen Monitor machen, damit keine Data Races auftreten.\\
\\
Das Setzen der Dokumente selbst können wir natürlich auch nebenläufig gestalten. Hierzu kann sich jeweils ein Thread um ein Blockelement kümmern. Es bleibt die Frage, wie viele Threads wir hierfür nutzen sollten, da ab einem bestimmten Punkt der Overhead dem Speed-Up überwiegt. Entweder könnte sich je ein Thread um je ein Element kümmern oder wir haben weniger Threads, die sich nach der fertigen Bearbeitung eines Blockelementes ein neues zugewiesen bekommen.\\
\\
Schließlich müssen wir noch unnötige Arbeit vermeiden. Das heißt, sobald der Parser oder einer der Threads bemerkt, dass das Dokument nicht setzbar ist, so sollen alle an ebenjenem arbeitenden Threads terminieren. Hierzu können wir den boolean $ unableToBreak $ nutzen, welcher entsprechend bei einem Error während der Blockelementverarbeitung gesetzt wird. Die arbeitenden Threads können dann immer checken (beispielsweise in der Bedingung einer While-Schleife) ob das Dokument noch setzbar ist und falls nicht terminieren. Offensichtlich braucht ebenjener Boolean also ein lock.\\
Sollte also der Parser beim parsen eine $ IOException $ werfen, so kann dessen Thread entsprechend jene Exception catchen und anschließend den Boolean $ unableToBreak $ setzen.\\\\
Aufgabe 2 / Bonus:\\
Hierfür müssen wir auch noch das Drucken nebenläufig zum Einlesen und Setzen implementieren. Dazu soll ein Thread immer wieder probieren, ob mit der aktuellen Anzahl an Pieces (fertige Absätze etc.) eine Seite fertig gesetzt werden kann (damit er dies nicht immer wieder unnötig versucht, kann er prüfen ob die Anzahl an pieces seit dem letzen Versuch größer geworden ist und entsprechend wird $ await() $ und $ signal() $ genutzt um Busy-Wait zu verhindern). Ob dies möglich ist, kann er dadurch sehen, ob die Länge der pages liste um eins länger als die aktuell zu druckende Seite ist (also z.b. Länge 2 wenn er die erste Seite drucken möchte).\\
Selbstverständlich kann er die letzte Seite drucken, sobald alle anderen Threads terminiert sind.\\
Sollte ebenjener Thread bemerken, dass das Dokument nicht setzbar ist, muss er äquivalent wie oben allen andren Threads ein Zeichen zur Terminierung geben um unnötiges Arbeiten zu verhindern.

%%%%%%%%%%%%%%%%%%%%%%%%%%%%%%%%%
 \newpage  % neuer Abschnitt auf neue Seite, kann auch entfallen
%%%%%%%%%%%%%%%%%%%%%%%%%%%%%%%%%
 
\section{Meilenstein $\Omega$}

  % Literaturverzeichnis (beginnt auf einer ungeraden Seite)
  \newpage

%\begin{thebibliography}{Lam00}
 
%\end{thebibliography}
 
      
  % ggf. hier Tabelle mit Symbolen 
  % (kann auch auf das Inhaltsverzeichnis folgen)

%\newpage
  
% \thispagestyle{empty}


%\vspace*{8cm}


%\section*{Erkl\"arung}

%Ich  versichere  wahrheitsgem\"a\ss,  die  Arbeit selbstst\"andig verfasst,  alle  benutzten  Hilfsmittel  vollst\"andig  und  genau  angegeben  und  alles kenntlich  gemacht  zu  haben,  was  aus  Arbeiten  anderer  unver\"andert  oder  mit  Ab\"anderungen entnommen  wurde,  sowie die Satzung  des  KIT  zur  Sicherung guter wissenschaftlicher Praxis in der jeweils g\"ultigen Fassung beachtet zu haben.
%\\[2ex] 

%\noindent
%Ort, den Datum\\[5ex]

% Unterschrift (handgeschrieben)



\end{document}

